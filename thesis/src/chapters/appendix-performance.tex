\chapter{Performance tests results}

The official documentation says that talloc is about 4 \% slower than
malloc\cite{TallocDoc}. However, it does not specify the testing environment and
measuring methodology. Therefore, I was eager to try it myself and I have
prepared several test cases: basic memory allocation, array allocation, memory
reallocation and string concatenation. 

\section{Environment}

Basic information:

\begin{table}[h]
  \begin{tabular}{l l}
  CPU: & Intel(R) Xeon(R) CPU W3550  @ 3.07GHz \\
  Main memory: & 2GB DDR3 ECC Synchronous 1333 MHz \\
  OS: & Fedora 15 x86_64 \\
  glibc: & glibc-2.14.1-6.x86_64 \\
  libtalloc: & libtalloc-2.0.6-1.fc15.x86_64
  \end{tabular}
\end{table}

\noindent
Dump of |lshw| program is stored on the attached DVD at |/documents/lshw|.

\section{Methodology}

I have used the system tap probes to retrieve the time needed to perform the
requested operations. These probes measure the time span between the enter to
and the return from the test function with microsecond precision. Each test case
has been performed for a hundred times. I have computed a median from all the
measured numbers to eliminate undesirable random fluctuations.

All the tests were originally run in the same binary. However, this approach was
generating invalid results, mainly in calloc and realloc tests -- they were
faster then malloc. And talloc equivalents that were executed at last were
faster then these tests. Probably the |glibc| is performing some sort of
optimization in these functions. Therefore, I have split the test into
separated programs to get the most accurate results.

To make the testing automatic and easy to reproduce, I have create a shell
script for each test. The output of this script is in the format of comma
separated values. The header line consists of names of the measured functions.
The values represent a time span in microseconds that was required to run the
function.

\section{Test: Memory allocation}

This test measures the time needed to allocate 100B for 100,000 times.

\funclistend
The source code of this test can be found on the attached DVD at
|/examples/src/performance/malloc|. The test can be run automatically with:
\begin{commandline}
sudo bash <dvd>/examples/scripts/perftest-malloc.sh > output_file.csv
\end{commandline}
\funclistend
The values I have measured are stored in the ODS file which is located on the
attached DVD medium at |/documents/results.ods|.

\subsection{Conclusion}

\section{Test: Array allocation}

This test measures the time needed to allocate 100 elements of |int| for 100,000
times.

\funclistend
The source code of this test can be found on the attached DVD at
|/examples/src/performance/calloc|. The test can be run automatically with:
\begin{commandline}
sudo bash <dvd>/examples/scripts/perftest-calloc.sh > output_file.csv
\end{commandline}
\funclistend
The values I have measured are stored in the ODS file which is located on the
attached DVD medium at |/documents/results.ods|.

\subsection{Conclusion}

\section{Test: Memory reallocation}

This test measures the time needed to reallocate a pointer for 100,000 times
where each iteration increases the memory occupied by this pointer by 100B.

\funclistend
The source code of this test can be found on the attached DVD at
|/examples/src/performance/realloc|. The test can be run automatically with:
\begin{commandline}
sudo bash <dvd>/examples/scripts/perftest-realloc.sh > output_file.csv
\end{commandline}
\funclistend
The values I have measured are stored in the ODS file which is located on the
attached DVD medium at |/documents/results.ods|.

\subsection{Conclusion}

\section{Test: String concatenation}

This test measures the time needed to concatenate an array that contains
100,000 strings where each string consists of ten characters.

\funclistend
The source code of this test can be found on the attached DVD at
|/examples/src/performance/concat|. The test can be run automatically
with:
\begin{commandline}
sudo bash <dvd>/examples/scripts/perftest-string-append.sh > output_file.csv
\end{commandline}
\funclistend
The values I have measured are stored in the ODS file which is located on the
attached DVD medium at |/documents/results.ods|.

\subsection{Conclusion}
