\chapter{Performance tests results}

The official documentation says that talloc is about 4 \% slower than
malloc\cite{TallocDoc}. However, it does not specify the testing environment and
measuring methodology. Therefore, I was eager to try it myself and I have
prepared several test cases: basic memory allocation, array allocation, memory
reallocation and string concatenation.

\section{Environment}

Basic information:

\begin{table}[h]
  \begin{tabular}{l l}
  CPU: & Intel(R) Xeon(R) CPU W3550  @ 3.07GHz \\
  Main memory: & 2GB DDR3 ECC Synchronous 1333 MHz \\
  OS: & Fedora 15 x86_64 \\
  glibc: & glibc-2.14.1-6.x86_64 \\
  libtalloc: & libtalloc-2.0.6-1.fc15.x86_64
  \end{tabular}
\end{table}

\noindent
Dump of |lshw| program is stored on the attached DVD at |/documents/lshw|.

\section{Methodology}

I have used the system tap probes to retrieve the time needed to perform the
requested operations. These probes measure the time span between the enter to
and the return from the test function with microsecond precision. Each test case
has been performed for a hundred times. I have computed a median from all the
measured numbers to eliminate undesirable random fluctuations.

All the tests were originally run in the same binary. However, this approach was
generating invalid results, mainly in calloc and realloc tests -- they were
faster then malloc. And talloc equivalents that were executed at last were
faster then these tests. Probably the |glibc| is performing some sort of
optimization in these functions. Therefore, I have split the test into
separated programs to get more accurate results.

\noindent
To make the testing automatic and easy to reproduce, I have created several
shell scripts. The output of these scripts is in the format of comma separated
values. The header line consists of names of the measured functions. The values
represent a time span in microseconds that was required to run the function.

\section{Test suite}

The test suite consists of:
\begin{itemize}
  \item Memory allocation (malloc)
  \item Array allocation (calloc)
  \item Memory reallocation (realloc)
  \item String concatenation (strcat)
\end{itemize}

\noindent
The source codes of these tests can be found on the attached DVD at
|<dvd>/examples/src/performance|.

\funclistend
The tests can be run separately using:
\begin{commandline}
sudo bash <dvd>/examples/scripts/perftest-malloc.sh \
     loops size pool-guess testcount > output_file.csv

sudo bash <dvd>/examples/scripts/perftest-calloc.sh \
     loops array-length testcount > output_file.csv

sudo bash <dvd>/examples/scripts/perftest-realloc.sh \
     loops increment testcount > output_file.csv

sudo bash <dvd>/examples/scripts/perftest-concat.sh \
     array-length string pool-guess testcount \
     > output_file.csv
\end{commandline}
\begin{itemize}
  \item |testcount| -- how many times is the test binary executed
  \item |loops| -- how many times is the tested function called
  \item |array-length| -- how many elements should be stored in the array
  \item |pool-guess| -- what size should be used for the |talloc_pool()|
  \item |string| -- each element of the array in the concatenation test is
  a duplication of this string
\end{itemize}

\noindent
The tests can be also executed all at once with the same parameters as I
have used with the following command:
\begin{commandline}
sudo bash <dvd>/examples/scripts/perftest-run.sh PREFIX
\end{commandline}
\funclistend
It stores the measured values in |<PREFIX><testcase>|.

\section{Conclusion}

The official speed test run on my computer says that the talloc is about 12 \%
slower than malloc. The results I have measured with my test suite cannot be
simply summarized on a few lines. Therefore, I have prepared a spreadsheet that
contains a lot of computed data and charts, which I believe are for this
case much better than words. The document is located at
|<dvd>/documents/results.ods|.

Statistically speaking, the talloc is much slower than the glibc allocator.
However, even when the tests use values that can be hardly achieved in most
applications, it still makes the difference only in miliseconds precision.
 