\section{Working with strings}
\label{talloc:sec:strings}

One of the most common issues that a programmer must take care of in a C
programming language is a duplication of and formatting a string.

In the C standard library this is usually done with |strdup()| (string
duplication), |strndup()| (string duplication with length limitation) and
|sprintf()| (stores the formatted string in a preallocated buffer).

The POSIX compliant systems provide us a much nicer function for creating a
formatted string -- |asprintf()| that allocates all the necessary space
automatically or its non-variadic equivalent |vasprintf()|.

Talloc has adopted its own variants of not only |strdup()| and |strndup()| from
the C standard library but also |asprintf()| and |vasprintf()| which are
demanded by the POSIX standard.

As an addition to these basic routines it adds a very useful alternatives that
will append the result to an existing string or at the end of the context. This
gives us a total of twelve functions for string manipulation.

All of these routines returns a string that is also a valid talloc context
which is named exactly the same as the output string. The basic variants take as
the first parameter an arbitrary talloc context which will serve as a parent to
the newly created context. The extended |_append| variants take a talloc context
that is also a valid string to which the result will be appended.

\subsection{Duplicating a string}

\begin{funcproto}
char* talloc_strdup(TALLOC_CTX *ctx, const char *str)
\end{funcproto}
\begin{funcdesc}
  Creates a new talloc context as a child of |ctx| and duplicates |str| to
  this context. Returns the new string.
\end{funcdesc}
\begin{funcproto}
char* talloc_strdup_append(char *dest, const char *str)
\end{funcproto}
\begin{funcdesc}
  Reallocates |dest| to required length and appends |str| to |dest|. Returns the
  new pointer to |dest| (the original pointer may not be longer valid).
\end{funcdesc}
\begin{funcproto}
char* talloc_strdup_append_buffer(char *dest, const char *s)
\end{funcproto}
\begin{funcdesc}
  This is similar to the previous funcion but it append |str| to the end of the
  context itself.
\end{funcdesc}
\funclistend
The |strndup| variants have the same behaviour but they will duplicate only |n|
characters from |str|.

\begin{funcproto}
char* talloc_strndup(TALLOC_CTX *ctx, const char *str,
                     size_t n)
char* talloc_strndup_append(char *dest, const char *str,
                            size_t n)
char* talloc_strndup_append_buffer(char *dest, const char
                                   *str, size_t n)
\end{funcproto}

\subsection{Formatted string}

Functions for formatted strings are very similar to the previous. The difference
is that they take a printf-style format string and its parameters. Talloc define
both variadic and non-variadic variants.

Variadic:
\begin{funcproto}
char* talloc_asprintf(TALLOC_CTX *ctx, const char *fmt, ...)
char* talloc_asprintf_append(char *dest, const char *fmt,
                             ...)
char* talloc_asprintf_append_buffer(char *dest, const char
                                    *fmt, ...)
\end{funcproto}

Non-variadic:
\begin{funcproto}
char* talloc_vasprintf(TALLOC_CTX *ctx, const char *fmt,
                       va_list ap)
char* talloc_vasprintf_append(char *dest, const char *fmt,
                              va_list ap)
char* talloc_vasprintf_append_buffer(char *dest, const char
                                     *fmt, va_list ap)
\end{funcproto}

\subsection{Examples}