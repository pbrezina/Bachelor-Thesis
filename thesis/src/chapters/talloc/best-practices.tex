\section{Best practices}
\label{talloc:sec:best-practices}

One of the purposes of talloc is to provide a much easier way of deallocating
data and lower the possibility of creating memory leaks. This is achieved by
hierarchical structure of talloc contexts when freeing one context will end up
freeing its descendants too.

\subsection{Every function should use it's own context}
\label{talloc:subsec:function-use-own-context}

\subsubsection{Functions that does not return any dynamically allocated value}

It is a good practice to create a temporary talloc context at the function
beginning and free this context just before the return statement. Every data
must be allocated on this context or on its children. This ensures that no
memory leaks are created.

\begin{lstlisting}[caption={Temporary context \#1},label=lst:tmp-ctx-1]
int bar()
{
  int ret;
  TALLOC_CTX *tmp_ctx = talloc_new(NULL);
  if (tmp_ctx == NULL) {
    ret = ENOMEM;
    goto done;
  }
  /* allocate data on tmp_ctx or on its descendants */
  ret = EOK;
done:
  talloc_free(tmp_ctx);
  return ret;
}
\end{lstlisting}

\subsubsection{Functions returning dynamically allocated values}

If our function returns any dynamically allocated value, its first parameter
should always be the destination talloc context. Inside this function we will
create a new, temporary context which is freed just before the function returns.
If everything goes well, we will steal all output variables from the temporary
context to the destination memory context.

This pattern ensures that if some error occurres (e.g. insufficient amount of
memory), all allocated data are freed and no garbage appears in the destination
context.

\begin{lstlisting}[caption={Temporary context \#2},label=lst:tmp-ctx-2]
int struct_foo_init(TALLOC_CTX *mem_ctx, struct foo **_foo)
{
  int ret;
  struct foo *foo = NULL;
  TALLOC_CTX *tmp_ctx = talloc_new(NULL);
  if (tmp_ctx == NULL) {
    ret = ENOMEM;
    goto done;
  }
  foo = talloc_zero(tmp_ctx, struct foo);
  /* ... */
  *_foo = talloc_steal(mem_ctx, foo);
  ret = EOK;
done:
  talloc_free(tmp_ctx);
  return ret;
}
\end{lstlisting}

\subsection{Allocate temporary contexts on NULL}
\label{talloc:subsec:tmp-ctx-on-null}

As you can see on the listing \ref{lst:tmp-ctx-2}, instead of allocating the
temporary context directly on |mem_ctx|, we created a new top level context
using |NULL| as the parameter for |talloc_new()| function. Take a look at the
following example.

\begin{lstlisting}[caption={Temporary context \#3},label=lst:tmp-ctx-3]
char * create_user_filter(TALLOC_CTX *mem_ctx,
                          uid_t uid, const char *username)
{
  char *filter = NULL;
  char *sanitized_username = NULL;
  /* tmp_ctx is a child of mem_ctx */
  TALLOC_CTX *tmp_ctx = talloc_new(mem_ctx);
  if (tmp_ctx == NULL) {
    return NULL;
  }
  
  sanitized_username = sanitize_string(tmp_ctx, username);
  if (sanitized_username == NULL) {
    talloc_free(tmp_ctx);
    return NULL;
  }
  
  filter = talloc_aprintf(tmp_ctx,"(|(uid=%llu)(uname=%s))",
                          uid, sanitized_username);
  if (filter == NULL) {
    return NULL; /* tmp_ctx is not freed */ (*@\label{lst:tmp-ctx-3:leak}@*)
  }
  
  /* filter becomes a child of mem_ctx */
  filter = talloc_steal(mem_ctx, filter);
  talloc_free(tmp_ctx);
  return filter;
}
\end{lstlisting}

We forgot to free |tmp_ctx| before the |return| statement on line
\ref{lst:tmp-ctx-3:leak}. However it is created as a child of |mem_ctx| context
and as such it will be freed as soon as the |mem_ctx| is freed so no detectable
memory leak is created.

On the other hand we do not have any way how to access the allocated data
and for what we know |mem_ctx| may exists for a lifetime of our application.
For these reasons this should be considered as a memory leak. But how can we
detect it when it is properly freed? The only way is to notice the mistake in
the source code.

But if we create the temporary context as a top level context, it will not be
freed and memory diagnostic tools
(e.g. |valgrind|\footnote{\url{http://valgrind.org}}) are able to do their job.

\subsection{Temporary contexts and talloc pool}
\label{talloc:subsec:tmp-ctx-and-pool}

If we want to take the advantage of the talloc pool but also keep to the
pattern introduced in previous section, we are unable to do it directly. The
best thing to do is to create a conditional build where we can decide how do we
want to create the temporary context. For example, we can create following
macros:

\begin{lstlisting}[caption={Conditional temporary context
macros},label=lst:tmp-ctx-4]
#ifdef USE_POOL_CONTEXT
  #define CREATE_POOL_CTX(ctx, size) talloc_pool(ctx, size)
  #define CREATE_TMP_CTX(ctx)        talloc_new(ctx)
#else
  #define CREATE_POOL_CTX(ctx, size) talloc_new(ctx)
  #define CREATE_TMP_CTX(ctx)        talloc_new(NULL)
#endif
\end{lstlisting}

Now if our application is under development, we will build it with macro
|USE_POOL_CONTEXT| undefined. This way, we  can use memory diagnostic
utilities for memory leaks detection.

The release vertions will be compiled with the macro defined. This will  enable
pool contexts and therefore reduce the |malloc()| calls, which will end up in a
little bit faster processing.

\begin{lstlisting}[caption={Conditional temporary context},label=lst:tmp-ctx-5]
int struct_foo_init(TALLOC_CTX *mem_ctx, struct foo **_foo)
{
  int ret;
  struct foo *foo = NULL;
  TALLOC_CTX *tmp_ctx = CREATE_TMP_CTX(mem_ctx);

  ...
}

errno_t handle_request(TALLOC_CTX mem_ctx)
{
  int ret;
  struct foo *foo = NULL;
  TALLOC_CTX *pool_ctx = CREATE_POOL_CTX(NULL, 1024);
  
  ret = struct_foo_init(mem_ctx, &foo);
  ...
}
\end{lstlisting}