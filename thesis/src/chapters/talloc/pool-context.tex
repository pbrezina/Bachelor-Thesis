\section{Decreasing number of malloc() calls}
\label{talloc:pool}

Allocation of a new memory is an expensive kernel space operation. Large
programs can contain thousands of calls of |malloc()| for a single computation.
This can result in an undesirable slowdown of the application. We can avoid
this slowdown by decreasing the number of |malloc()| calls by using a memory
pool.

Memory pool\footnote{An example of a simple memory pool can be found in appendix
@TODO} is a preallocated memory space with a fixed size. If we need to allocate
new data we will take the desired amount of the memory from the pool instead of
requesting new memory from the system. This is done by creating a pointer that
points inside the preallocated memory. Such pool must not be reallocated as it
would change its location -- pointers that were pointing inside the pool would
become invalid. Therefore, a memory pool requires a very good estimate of the
required memory space.

The talloc library contains its own implementation of a memory pool. It is
highly transparent for the programmer. The only thing that needs to be done is
an initialization of a new pool context using |talloc_pool()|\footnotemark --
it can be used in the same way as any other context.

\footnotetext{\lstinline{TALLOC_CTX *talloc_pool(TALLOC_CTX *ctx, size_t size)}}

Refactoring of an existing code (that uses talloc) to take the advantage of a
memory pool is quite simple due to the following properties of the pool context:

\begin{itemize}
  \item if we are allocating data on a pool context, it takes the desired
  amount of memory from the pool,
  \item if the context is a descendant of the pool context, it takes the space
  from the pool as well,
  \item if the pool does not have sufficient portion of memory left, it will
  create a new non-pool context, leaving the pool intact,
\end{itemize}

\begin{lstlisting}[caption={Talloc pool},label=lst:talloc_pool,
morekeywords={talloc_pool}]
/* allocate 1KiB in a pool */
TALLOC_CTX *pool_ctx = talloc_pool(NULL, 1024);

/* take 512B from the pool, 512B is left there */
void *ptr = talloc_size(pool_ctx, 512);

/* 1024B > 512B, this will create new talloc chunk outside
   the pool */
void *ptr2 = talloc_size(ptr, 1024);

/* the pool still contains 512 free bytes
 * this will take 200B from them */
void *ptr3 = talloc_size(ptr, 200);

/* this will destroy context 'ptr3' but the memory
 * is not freed, the available space in the pool
 * will increase to 512B */
talloc_free(ptr3);

/* this will free memory taken by 'pool_ctx'
 *  and 'ptr2' as well */
talloc_free(pool_ctx);
\end{lstlisting}

\noindent
This is very nice. However, it has one big issue that we must be aware of. See,
if we change the parent of a child of talloc pool to a parent that is outside
this pool, the whole pool memory will not be freed until the child is freed.
For this reason we must be very careful when stealing a descendant of a pool
context.

\begin{lstlisting}[caption={Stealing from pool context},
morekeywords={talloc_pool,talloc_steal,talloc_free}]
TALLOC_CTX *mem_ctx = talloc_new(NULL);
TALLOC_CTX *pool_ctx = talloc_pool(NULL, 1024);
struct foo *foo = talloc(pool_ctx, struct foo);

/* mem_ctx is not in the pool */
talloc_steal(mem_ctx, foo);

/* pool_ctx is marked as freed but the memory is not
   deallocated, accessing the pool_ctx again will cause
   an error */
talloc_free(pool_ctx);

/* now is pool_ctx deallocated */
talloc_free(mem_ctx);
\end{lstlisting}

\noindent
It may be often better to copy the memory we want to steal to avoid this
problem. We can use |talloc_memdup()| to do this:

\begin{funcproto}
TALLOC_CTX *talloc_memdup(TALLOC_CTX *parent,
                          TALLOC_CTX *ctx, size_t size)
\end{funcproto}
\begin{funcdesc}
Creates a new context and copies |size| bytes from |ctx| into it. The name of
the context will be the location in the source code where is this function
called.
\end{funcdesc}
\funclistend
For example, the most effective way of creating the LDAP filter with groups
(introduced in the Listing \ref{lst:append_buffer}) would be:

\begin{lstlisting}[caption={Appending a string with a pool context},
morekeywords={talloc_pool,talloc_memdup}]
char* create_filter(TALLOC_CTX *mem_ctx, char **groups)
{
  int i;
  TALLOC_CTX *pool_ctx = talloc_pool(mem_ctx, 10240);
  char *output = NULL;
  char *filter = talloc_strdup(pool_ctx, "(|");
  for (i=0; groups[i] != NULL; i++) {
    filter = talloc_asprintf_append_buffer(
               filter, "(group=%s)", groups[i]);
  }
  filter = talloc_strdup_append_buffer(filter, ")");
  
  /* copy the filter */
  output = talloc_memdup(mem_ctx, filter,
                         talloc_array_length(filter));
  talloc_free(pool_ctx);

  return output;
}
\end{lstlisting}