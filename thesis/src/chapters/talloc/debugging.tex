\section{Debugging options}
\label{talloc:sec:debugging}

Although talloc makes memory management significantly easier than the standard
library does, developers are still only humans. Therefore it may be a good idea
to know the tools for the inspection of talloc memory usage.

\subsection{Talloc log and abort}

We have already encountered with the abort function in section
\ref{talloc:dyn-ts}. In that case it was used when a type mismatch was
detected. However talloc calls this abort function in several more situations:

\begin{itemize}
  \item when the provided pointer is not a valid talloc context,
  \item when the meta data are invalid -- probably due to a memory corruption,
  \item and when access after free is detected.
\end{itemize}

The third one is probably the most interesting. It can help us with detecting
double free a context or any other manipulation with it via talloc functions
(using it as a parent, stealing it, etc.).

Before the context is freed talloc sets a flag bit in the meta data. This is
then used to detect the access after free. It basically works on the assumption
that the memory will stay unchanged (at least for a while) even when it is
properly deallocated. This will work event if the memory is filled with the
value specified in |TALLOC_FREE_FILL| environment variable, because it will
fill only the data part and leaves the meta data intact.

Apart from the abort function, talloc uses a log function to provide additional
information to the aforementioned violations. To enable logging we shall set the
log function with:

\begin{funcproto}
void talloc_set_log_fn(void (*fn)(const char *message))
void talloc_set_log_stderr()
\end{funcproto}
\funclistend
The latter one is a shortcut for a common scenario -- it will print the
|message| in the standard error stream.

Here is an sample output of accessing a context after it has been freed:

\begin{lstlisting}[caption={Talloc error -- access after free}]
talloc_set_log_stderr();
TALLOC_CTX *ctx = talloc_new(NULL);

talloc_free(ctx);
talloc_free(ctx);

results in:
talloc: access after free error - first free may be at ../src/main.c:55
Bad talloc magic value - access after free
\end{lstlisting}

\noindent
And here is an example of the invalid context:

\begin{lstlisting}[caption={Talloc error -- not a talloc context}]
talloc_set_log_stderr();
TALLOC_CTX *ctx = talloc_new(NULL);
char *str = strdup("not a talloc context");
talloc_steal(ctx, str);

results in:
Bad talloc magic value - unknown value
\end{lstlisting}

\subsection{Memory usage reports}

Talloc can print reports of memory usage of specified talloc context to a file
(or to |stdout| or |stderr|). The report can be simple or full. The simple
report provides information only about the context itself and its direct
descendants. The full report goes recursively through the entire context tree.

\begin{funcproto}
void talloc_report(TALLOC_CTX *ctx, FILE *file)
void talloc_report_full(TALLOC_CTX *ctx, FILE *file)
\end{funcproto}
\funclistend
We will use following code to retrieve the sample report:

\begin{lstlisting}[caption={Full report}]
struct foo {
  char *str;
};

TALLOC_CTX *ctx = talloc_new(NULL);
char *str =  talloc_strdup(ctx, "my string");
struct foo *foo = talloc_zero(ctx, struct foo);
foo->str = talloc_strdup(foo, "I am Foo");
char *str2 = talloc_strdup(foo, "Foo is my parent");

/* print full report */
talloc_report_full(ctx, stdout);
\end{lstlisting}

\noindent
It will print a full report of |ctx| to the standard output. The message should
be similar to:

\begin{lstlisting}[keywordstyle=,caption={Sample output of talloc full report}]
full talloc report on 'talloc_new: ../src/main.c:82' (total 46 bytes in 5 blocks)
  struct foo contains 34 bytes in 3 blocks (ref 0) 0x1495130
    Foo is my parent contains 17 bytes in 1 blocks (ref 0) 0x1495200
    I am Foo contains 9 bytes in 1 blocks (ref 0) 0x1495190
  my string contains 10 bytes in 1 blocks (ref 0) 0x14950c0
\end{lstlisting}

\noindent
We can notice in this report that something is wrong with the context containing
|struct foo|. We know that the structure has only one string element. However we
can see in the report that it has two children. This indicates that we have
either violated the memory hierarchy or forgot to free it as a temporary data.
Looking into the code, we can see that |"Foo is my parent"| should be attached
to |ctx|.

The following functions affect the behaviour of talloc reports. To get correct
results, they should be called before any other routine from talloc library.

\begin{funcproto}
void talloc_enable_null_tracking(void)
void talloc_disable_null_tracking(void)
\end{funcproto}
\begin{funcdesc}
Enables/disables tracking of talloc contexts without a parent. If it is enabled,
we can use |NULL| as a context for report and it will print the summary of all
available contexts.
\end{funcdesc}

\begin{funcproto}
void talloc_enable_leak_report(void)
void talloc_enable_leak_report_full(void)
\end{funcproto}
\begin{funcdesc}
  Enables automatic simple/full report of all contexts at program exit. The
  report is printed in |stderr|.
\end{funcdesc}

\subsection{Detecting memory leaks}

We frequently need to provide a talloc context as a parameter to a function, so
this function can use it as a parent context for output values. But sometimes a
programmer makes a mistake and allocates something more on the output context
than he is supposed to.

This issue can not be usually seen in talloc reports, but fortunately it can be
tested automatically with the knowledge of total size (size of a context and
its children) of the parent context.

\begin{lstlisting}[caption={Memory leaks detection \#1},
morekeywords={talloc_total_size,talloc_free}]
/* creates new talloc context *_foo as a child of mem_ctx */
int struct_foo_init(TALLOC_CTX *mem_ctx, struct foo **_foo);

int ret;
size_t total_size;
struct foo *foo = NULL;
TALLOC_CTX *ctx = talloc_new(NULL);

total_size = talloc_total_size(ctx)
ret = struct_foo_init(test_ctx, &foo);
talloc_free(foo);
if (total_size != talloc_total_size(ctx)) {
  /* struct_foo_init() allocated on 'ctx'
     more than 'foo' */
}
\end{lstlisting}

However such checks are not suitable for an application itself as we have to
free |foo| before other usage of the parent context. Therefore the proper place
for this is in your test suit, possibly in unit tests.

The SSSD uses a set of functions\footnote{Available in appendinx @TODO} which
goal is to detect memory leaks on a talloc context in unit tests created with
the Check\footnote{\url{http://check.sourceforge.net}} library (but it can be
easily modified for other test suits). It does it in a more advance manner than
the previous example introduced. It works on a push/pop basis, where the unit
test will fail if a context has after the pop operation different size than it
had during the push operation.

\begin{lstlisting}[caption={Memory leaks detection \#2},
morekeywords={leak_check_setup,check_leaks_push,
              check_leaks_pop,leak_check_teardown}]
/* creates new talloc context *_foo as a child of mem_ctx */
int struct_foo_init(TALLOC_CTX *mem_ctx, struct foo **_foo);

/* create new unit test with Check library */
START_TEST(test_struct_foo_init)
{
  int ret;
  struct foo *foo = NULL;
  TALLOC_CTX *test_ctx = talloc_new(NULL);
  fail_if(test_ctx == NULL); /* if the condition is met,
                                the unit test will fail */
  
  /* initialize the leak check routines */
  leak_check_setup();
  
  /* remember current test_ctx total size */
  check_leaks_push(test_ctx);
  
  ret = struct_foo_init(test_ctx, &foo);
  fail_if(ret != EOK)
  
  /* ... test foo values ... */
  
  talloc_free(foo);
  
  /* check for memory leak,
     this will fail the test if struct_foo_init()
     allocated more than 'foo' on test_ctx */
  check_leaks_pop(test_ctx);
  
  talloc_free(test_ctx);
  
  /* clean up after the leak check routines */
  leak_check_teardown();(*@\footnotemark@*)
}
\end{lstlisting}
\footnotetext{In the Check library, setup and teardown functions can be run
automatically with \lstinline{tcase_add_}\lstinline{checked_fixture()}}
