\section{Using destructors}
\label{talloc:sec:destructors}

Destructors are well known methods in the world of object oriented programming.
A destructor is a method of an object that is automatically run when the object
is destroyed. It is usually used to return resources taken by the object back to
the system (e.g. closing file descriptors, terminating connection to a database,
deallocating memory).

With talloc we can take the advantage of destructors even in C. We can easily
attach our own destructor to a talloc context. When the context is freed, the
destructor is run automatically.

\subsection{Destructor prototype}

Destructor in talloc is a function with following prototype:

\begin{funcproto}
int destructor(void *ctx)
\end{funcproto}
\funclistend
The parameter |ctx| is the talloc context that is about to be freed. The proper
type can be retrieved with |talloc_get_type()|. The destructor is type safe during
the compilation if we use |GCC 3| or newer (it uses the |__typeof__|
feature of this compiler). With this compiler we can avoid 
using |void*| and the parameter can be any valid pointer type.

The destructor should return |0| for succes and |-1| on failure. If it returns
|-1| then |talloc_free()| will fail and the memory of the context and its
children will not be freed.

\subsection{Managing destructors}

To attach a destructor to a talloc context we will use:

\begin{funcproto}
void talloc_set_destructor(TALLOC_CTX *ctx,
                           int(*)(void *) destructor) 
\end{funcproto}
\funclistend
Every context can have one destructor attached as a maximum. If we call this
function more than once on the same context, the current destructor will be
overwritten. To dettach the destructor from the context we will simply set
|NULL| as the new destructor.

\subsection{Children contexts with destructors}

If we will attempt to free a talloc context (|ctx|) that has one or more
children with a destructor attached, we may encounter a strange behaviour. If a
destructor of |ctx| fails, |talloc_free(ctx)| will fail as well and there will
be no attempt to free the children of |ctx|.

But if the destructor of |ctx| succeeds, |talloc_free(ctx)| will succeed and
the memory of |ctx| is deallocated even if destructor on one of the children
fails.

\subsection{Examples}

\begin{lstlisting}[caption={Close a file}]
int destroy_fd(void *ctx)
{
  int *fd = talloc_get_type_abort(ctx, int);
  close(*fd);
  return 0;
}

int* open_file(TALLOC_CTX *mem_ctx,
                const char *filename,
                int flags)
{
  int *fd = talloc_ptrtype(mem_ctx, fd);
  if (file == NULL) {
    return NULL;
  }
  *fd = open(filename, flags);
  if (*fd < 0) {
    talloc_free(fd);
    return NULL;
  }
  
  talloc_set_destructor(fd, destroy_fd);
  return fd;
}
\end{lstlisting}