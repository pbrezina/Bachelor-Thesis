\section{Using destructors}
\label{talloc:sec:destructors}

Destructors are well known methods in the world of object oriented programming.
A destructor is a method of an object that is automatically run when the object
is destroyed. It is usually used to return resources taken by the object back to
the system (e.g. closing file descriptors, terminating connection to a database,
deallocating memory).

With talloc we can take the advantage of destructors even in C. We can easily
attach our own destructor to a talloc context. When the context is freed, the
destructor is run automatically.

\subsection{Destructor prototype}

Destructor in talloc is a function with following prototype:

\begin{funcproto}
int destructor(void *ctx)
\end{funcproto}
\funclistend
The parameter |ctx| is the talloc context that is about to be freed. The proper
type can be retrieved with |talloc_get_type()|. The destructor is type-safe
during the compilation if we use |GCC 3| or newer (it uses the |__typeof__|
feature of this compiler). With this compiler we can avoid using |void*| and the
parameter can be any valid pointer type.

The destructor should return |0| for succes and |-1| on failure. If it returns
|-1| then |talloc_free()| fails and the memory of the context and its children
is not deallocated.

\subsection{Managing destructors}

The following function is used to attach a destructor to a talloc context:

\begin{funcproto}
void talloc_set_destructor(TALLOC_CTX *ctx,
                           int(*)(void *) destructor) 
\end{funcproto}
\funclistend
Every context can have one destructor attached as a maximum. If the function is
called more than once on the same context, the current destructor is
overwritten. To detach the destructor from the context we shall simply set
|NULL| as the new destructor.

\subsection{Children contexts with destructors}

If we attempt to free a talloc context (|ctx|) that has one or more children
with a destructor attached, we may encounter a strange behaviour. If a
destructor of |ctx| fails, |talloc_free(ctx)| fails as well and there will
be no attempt to free the children of |ctx|.

But if the destructor of |ctx| succeeds, |talloc_free(ctx)| succeeds too and
the memory of |ctx| is deallocated even if destructor on one of the children
fails.

\subsection{Example}

Imagine that we have a dynamically created linked list. We do not want to use an
element of the list after it is removed from the list. Therefore, removing the
element would mean also deallocating it. We can do this at once by setting a
destructor on the element which will remove it from the list and
|talloc_free()| will do the rest.

The destructor would be:

\begin{lstlisting}[caption={Remove an element from the list -- destructor}]
int list_remove(void *ctx)
{
    struct list_el *el = NULL;
    el = talloc_get_type_abort(ctx, struct list_el);
    /* remove element from the list */    
}
\end{lstlisting}

\noindent
The |GCC 3| or newer can check for the types during the compilation. So if it is
our major compiler, we can use a little bit nicer destructor:

\begin{lstlisting}[caption={Remove an element from the list -- type-safe
destructor}]
int list_remove(struct list_el *el)
{
    /* remove element from the list */    
}
\end{lstlisting}

\noindent
Now we will assign the destructor to the list element. We can do this directly
in the function that inserts it.

\begin{lstlisting}[caption={Remove an element from the list when freed},
morekeywords={talloc_set_destructor}]
struct list_el* list_insert(TALLOC_CTX *mem_ctx,
                            struct list_el *where,
                            void *ptr)
{
  struct list_el *el = talloc(mem_ctx, struct list_el);
  el->data = ptr;
  /* insert into list */
  
  talloc_set_destructor(el, list_remove);
  return el;
}
\end{lstlisting}

\noindent
Because talloc is a hierarchical memory allocator, we can go a step further and
free the data with the element as well:

\begin{lstlisting}[caption={Free the data with the list element},
morekeywords={talloc_steal}]
struct list_el* list_insert_free(TALLOC_CTX *mem_ctx,
                                 struct list_el *where,
                                 void *ptr)
{
  struct list_el *el = NULL;
  el = list_insert(mem_ctx, where, ptr);

  talloc_steal(el, ptr);

  return el;
}
\end{lstlisting}