\section{Stealing a context}
\label{talloc:stealing}

Talloc has the ability to change the parent of a talloc context to another
context. This is necessary if we want the pointer to outlive the context it is
created on.

This is frequently used in functions that return data which has been created on
an inner context. The common usage of this technique is described in section
\cmplref{talloc:subsec:function-use-own-context}.

\begin{funcproto}
void *talloc_steal(TALLOC_CTX *ctx, const void *ptr)
\end{funcproto}
\begin{funcdesc}
  Changes the parent of the |ptr| to |ctx| and returns |ptr|.
\end{funcdesc}
\begin{funcproto}
void *talloc_move(TALLOC_CTX *ctx, const void **ptr)
\end{funcproto}
\begin{funcdesc}
  This function is a wrapper around |talloc_steal()|. It will assing |NULL| into
  |*ptr| after the parent of |ptr| is changed. Returns the same pointer as
  |talloc_steal()| would return.
\end{funcdesc}
