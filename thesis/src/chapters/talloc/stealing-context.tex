\section{Stealing a context}
\label{talloc:stealing}

Talloc has the ability to change the parent of a talloc context to another
one. This operation is commonly referred to as stealing and it is one of
the most important actions performed with talloc contexts.

Stealing a context is necessary if we want the pointer to outlive the context it
is created on. This has many possible use cases, for instance stealing a result
of a database search to an in-memory cache context, changing the parent of a
field of a generic structure to a more specific one or wise versa. The most
common scenario, at least in SSSD, is to steal output data from a function-specific
context to the output context given as an argument of that function -- this is
more deeply explained as one of the best practices in Section 
\ref{talloc:subsec:function-use-own-context}.

\begin{funcproto}
void *talloc_steal(TALLOC_CTX *ctx, const void *ptr)
\end{funcproto}
\begin{funcdesc}
  Changes the parent of the |ptr| to |ctx| and returns |ptr|.
\end{funcdesc}
\begin{funcproto}
void *talloc_move(TALLOC_CTX *ctx, const void **ptr)
\end{funcproto}
\begin{funcdesc}
  Additionally assigns |NULL| into |ptr|.
\end{funcdesc}
\funclistend
In general, the pointer itself will not be changed (it will only replace the
parent in the meta data). But the common usage is that we will assign the
result to another variable, so we should avoid accessing the pointer from the
original variable if it is not necessary. Therefore |talloc_move()| is a
preferred way of stealing a context as it protects the pointer from being
accidentally freed and accessed using the old variable after its parent have
been changed.

\begin{lstlisting}
struct foo *foo = talloc_zero(mem_ctx, struct foo);

/* change parent of foo from mem_ctx to bar */
bar->foo = talloc_steal(bar, foo);

/* do the same but and assign foo = NULL */
bar->foo = talloc_move(bar, &foo);
\end{lstlisting}
