\section{Arrays and talloc}
\label{talloc:sec:arrays}

Another difficult thing in C language are dynamically allocated arrays. I can
not say that talloc make the creation of an array significantly easier, but it
makes the interface to look more natural. However, one of the talloc strongest
points is in the deallocation of arrays.

\subsection{Creating a new array}

To create an array, we can use several functions that are very similar to those
that allocates the basic types. In fact, those functions are more or less a
wrapper around the more generic ones where the size of the context is simply
calculated as |size * count| (as we would if we used the standard library).

\begin{funcproto}
(#type)* talloc_array(TALLOC_CTX *ctx, #type,
                   unsigned count)
(#type)* talloc_zero_array(TALLOC_CTX *ctx, #type
                           unsigned count)
void* talloc_array_size(TALLOC_CTX *ctx, size_t size,
                        unsigned count)
void* talloc_array_ptrtype(TALLOC_CTX *ctx, #ptr,
                           unsigned count)
\end{funcproto}

\subsection{Reallocating an array}

Talloc developers did not forget on reallocation either. It works the same way
as standard |realloc()| would, that is it may move the memory occupied by |ptr|
to a new location and returns the new pointer.

\begin{funcproto}
void* talloc_realloc(TALLOC_CTX *ctx, void *ptr,
                     #type, size_t count)
void* talloc_realloc_size(TALLOC_CTX *ctx, void *ptr,
                          size_t size)
\end{funcproto}

\subsection{Determining length of an array}

We can find ourselves in situation when we need to iterate through an array but
we are not able to use a special delimiter to mark the end of the array. We need
to have the knowledge of the size of this array stored in other variable.

Every talloc context carries its size in bytes with it, this gives us the means
to determine the length of the array at any time. Therefore we do not have to
carry the number of the elements separately. Talloc has defined a macro
|talloc_array_length(ctx)| for this purpose.

It can be also used as a more efficient way of computing a number of characters
in a string that has been created with talloc library.

\subsection{Deallocating arrays}

If we want to deallocate an array using the means of C standard library, we need
to free every element of the array individually in the first place. This usually
requires a lot of code (the more the bigger the dimension of the array is), not
to mention a lot of places for a mistake. But if we use talloc and we project
the array structure to the context hierarchy, we can do it very simply by
calling |talloc_free()| on the array.

\subsection{Examples}

\begin{lstlisting}[caption={Length of an array},label=lst:array-length]
TALLOC_CTX *ctx = talloc_new(NULL);
int *i = talloc_zero(ctx, int);
char *str = talloc_strdup(ctx, "hello world");
int *array = talloc_array(ctx, int, 10);

printf("%lu\n", talloc_array_length(ctx));  /*  0 */
printf("%lu\n", talloc_array_length(i));     /*  1 */
printf("%lu\n", talloc_array_length(str));   /* 12 */
printf("%lu\n", talloc_array_length(array)); /* 10 */

talloc_free(ctx);
\end{lstlisting}