\section{Talloc context}
\label{talloc:sec:context}

Talloc context is a logical unit of this memory pool system. In combination with
other contexts it creates an n-ary tree.

It is internally handled as a special structure (called talloc_chunk) that
contains information about the allocated data - name, size, destructor,
references, child contexts and its parent context. All of these information can
be read or manipulated with using proper functions.

Probably the biggest advantage over the other memory pool systems is that there
is no difference between a talloc context and a talloc pointer. Every pointer
that is returned from this library is considered to be a context and can be used
in such manner.

Talloc context is of type |TALLOC_CTX*|. However, |TALLOC_CTX| is only an alias
for |void| and exists only for a semantical reason - so we can differentiate
between |void*| (arbitrary data) and |TALLOC_CTX*| (talloc context).

\subsection{Creating a new context}
\label{talloc:subsec:new-context}

Many functions exist that has the ability to create a new talloc context. This
section describes only the most fundamental ones that deals with primitive data
types and structures. Functions that manipulates with strings and arrays are
described later in this text (sections \ref{talloc:sec:strings} and
\ref{talloc:sec:arrays}).

We can separate these functions to four categories: those that allows us to
set a name of the context, functions that creates a zero-length context and type
safe and type unsafe functions.

All of these functions share the following properties:
\begin{itemize}
  \item it returns new talloc context or |NULL| if the system is out of memory,
  \item the first parameter is a talloc context which serves as a parent of
  the new context,
  \item the parent context can be either an existing context or |NULL| which
  will create a new top level context.
\end{itemize}

\subsubsection{Contexts with custom name}

Every talloc context has a name.
If we want to create a context with a strong semantical meaning, we can set a
custom name to differentiate it from the other contexts.

% \begin{description}
%   \item[void * talloc_named(TALLOC_CTX *ctx, size_t size, const char *fmt,
%   \ldots)] \hfill  \\
%   This function creates a new talloc context of size |size|. This context will
%   be named according to specified |printf|-style format |fmt|.
% 
% \item[]
%   \item[void * talloc_named_const(TALLOC_CTX *ctx, size_t size, const char
%   *name)] \hfill \\
% 
% \end{description}

\subsubsection{Zero-length contexts}

Zero-length context is basically a context without any special semantical
meaning. We can use it the same way as any other context, the only difference is
that it does not contain any data and therefore it is strictly of type
|TALLOC_CTX*|. It is often used when we want to aggregate several data
structures under one parent (zero-length) context.

% \begin{description}
%   \item[TALLOC_CTX * talloc_init(const char *fmt, \ldots)] \hfill  \\
%   \item[TALLOC_CTX * talloc_new(TALLOC_CTX *ctx)] \hfill  \\
% \end{description}

\subsubsection{Type safe functions that creates a new context}

Type safe functions takes as one of their parameters a data type we want to
create. It allocates the size that is necessary for this type and returns a new
properly casted pointer. This is useful if we want to rely on the compiler to
detect type mismatches.

% \begin{description}
%   \item[(\#type) * talloc(TALLOC_CTX *ctx, \#type)] \hfill  \\
%   \item[(\#type) * talloc_zero(TALLOC_CTX *ctx, \#type)] \hfill  \\
% \end{description}

\subsubsection{Type unsafe functions that creates a new context}

Type unsafe functions takes as a parameter directly the size we want to
allocate instead of the type and returns a pointer to |void|. We should not use
these functions unless it is necessary (e.g. reading binary data from a file).

% \begin{description}
%   \item[void * talloc_size(TALLOC_CTX *ctx, size_t size)] \hfill  \\
%   \item[void * talloc_zero_size(TALLOC_CTX *ctx, size_t size)] \hfill  \\
%   \item[void * talloc_ptrtype(TALLOC_CTX *ctx, \#type)] \hfill  \\
% \end{description}

%- talloc_new
%- talloc
%- talloc_init
%- talloc_named
%- talloc_named_const
%- talloc_ptrtype

\subsection{Freeing a context}
\label{talloc:subsec:free-context}

There are two functions defined that deal with deallocating a context. Both
take memory context as their argument and if this context is |NULL| then no
action is performed.

\begin{funcproto}
int talloc_free(TALLOC_CTX *ctx)
\end{funcproto}
\begin{funcdesc}
  Deallocates memory occupied by the context and recursively frees its 
  children as well. The returned value is |0| on success, |-1| if |ctx| is NULL
  or if the destructor\footnote{More information on destructors is in section
  \cmplref{talloc:sec:destructors}} attached to this context fails.
\end{funcdesc}
\begin{funcproto}
void talloc_free_children(TALLOC_CTX *ctx)
\end{funcproto}
\begin{funcdesc}
  Frees only the children of the context.
\end{funcdesc}
\funclistend
Besides these two functions we can find useful a macro named |TALLOC_FREE| which
is defined as:

\begin{lstlisting}[caption={TALLOC_FREE(ctx)},label=lst:TALLOC_FREE]
#define TALLOC_FREE(ctx) do { \
  talloc_free(ctx);           \
  ctx = NULL;                 \
} while(0);
\end{lstlisting}

%- talloc_autofree_context
%- TALLOC_FREE
%- talloc_free - 0 or 1 parent
%- talloc_free_children
%- talloc_unlink - more parents

%-  talloc_set_destructors

\subsection{Changing parent of a context}
\label{talloc:subsec:stealing}

Talloc has the ability to change the parent of a talloc pointer to another
context. This is necessary if we want the pointer to outlive the context it is
created on.

This is frequently used in functions that return data which has been created on
an inner context. The common usage of this technique is described in section
\cmplref{talloc:subsec:function-use-own-context}.

\begin{funcproto}
void *talloc_steal(TALLOC_CTX *ctx, const void *ptr)
\end{funcproto}
\begin{funcdesc}
  Changes the parent of the |ptr| to |ctx| and returns |ptr|.
\end{funcdesc}
\begin{funcproto}
void *talloc_move(TALLOC_CTX *ctx, const void **ptr)
\end{funcproto}
\begin{funcdesc}
  This function is a wrapper around |talloc_steal()|. It will assing |NULL| into
  |*ptr| after the parent of |ptr| is changed. Returns the same pointer as
  |talloc_steal()| would return.
\end{funcdesc}
\
\subsection{Dynamic type checking}
\label{talloc:subsec:type-checking}

%- talloc_reparent
%- talloc_steal
%- talloc_move

\subsection{Managing references}
\label{talloc:subsec:references}

\subsection{Decreasing number of malloc() calls}
\label{talloc:subsec:pool}

%Allocation of a new memory space is usually an expensive kernel space
%operation.

Allocation of a new memory is an expensive kernel space operation and large
programs can contain thousands of calls of |malloc()| for a single operation.
This can pour out into an unnecessary slowdown of the application. Number of
|malloc()| calls can be decreased by using memory pool.

Memory pool\footnote{An example of a simple memory pool can be found in appendix
@TODO} is a preallocated memory space of a fixed size. If we need to allocate
new data we can simply take the desired amount from the pool instead of
requesting new memory from the system. Hovewer, this requires some special
behaviour of a programmer.

The talloc library contains its own implementation of a memory pool and the
great thing about this is that it is completely transparent for the programmer.
The only thing that needs to be done is an initialization of a new pool context
using |talloc_pool()|\footnote{\lstinline{TALLOC_CTX *
talloc_pool(TALLOC_CTX *ctx, size_t size)}}. This context can be used in the
same way as any other context.

Talloc pool context has the following properties:

\begin{itemize}
  \item if we are allocating data on a pool context, it takes the  desirable
  amount of space from the pool,
  \item if the context is a descendant of the pool context, it takes the space
  from the pool as well,
  \item if the pool is out of memory, it creates a new, normal context,
  \item if we change the parent of a child of talloc pool to a parent that is
  outside of this pool, the whole pool memory will not be freed until the child
  is freed.
\end{itemize}

\begin{lstlisting}[caption={Talloc pool},label=lst:talloc_pool]
/* allocate 1KiB in a pool */
TALLOC_CTX *pool_ctx = talloc_pool(NULL, 1024);

/* take 512B from the pool, 512B left */
void *ptr = talloc_size(pool_ctx, 512);

/* 1024B > 512B, this will create new talloc chunk */
void *ptr2 = talloc_size(ptr, 1024);

/* the pool still contains 512 free bytes
 * this will take 200B from them */
void *ptr3 = talloc_size(ptr, 200);

/* this will destroy context 'ptr3' but the memory
 * is not freed, the available space in the pool
 * will increase to 512B */
talloc_free(ptr3);

/* this will free memory taken by 'pool_ctx'
 *  and 'ptr3' as well */
talloc_free(pool_ctx);
\end{lstlisting}

%- talloc_get_type
%- talloc_get_type_abort