\section{Talloc context}
\label{talloc:sec:context}

Talloc context is the most important part of this library for it is responsible
for every single feature of this memory allocator. It is a logical unit which
represents a memory space managed by talloc.

From the programmer point of view, talloc context is completely equivalent to a
pointer that would return memory routines from the C standard library. This
means that every context that is returned from the talloc library can be used
directly in functions that do not use talloc internally. For example we can do
this:

\begin{lstlisting}
char *str1 = strdup("I am NOT a talloc context");
char *str2 = talloc_strdup(NULL, "I AM a talloc context");

printf("%d\n", strcmp(str1, str2) == 0);

free(str1);
talloc_free(str2); /* we can not use free(str2) */
\end{lstlisting}

This is possible because internally is the context handled as a special
fixed-length structure called talloc chunk. Each chunk stores context meta data
followed by memory space requested by the programmer. When talloc function
returns a context (pointer), it returns in fact a pointer to the user space
part of the chunk. And when we want to manipulate with this context, the
library will transform the pointer to the user space part back to the starting
address of the chunk. This is also the reason why we were unable to use
|free(str2)| in the previous example -- because  |str2| does not point at the
beginning of the allocated block of memory. This is illustrated on Figure
\ref{fig:talloc-context}.

\begin{figure}[H]
  \centering
  \setlength{\unitlength}{1cm}
\begin{picture}(4,7)
  % allocated memory box
  \put(0,7){\line(1,0){4}}
  \put(0,0){\line(1,0){4}}
  \put(0,0){\line(0,1){7}}
  \put(4,0){\line(0,1){7}}
  \put(0,0){\makebox(4,4){requested memory}}

  % context size
  \put(4.6,0){\vector(0,1){4}}
  \put(4.6,4){\vector(0,-1){4}}
  \put(4.8,0){\makebox(0,4)[l]{context size}}

  % allocated block
  \put(4.3,0){\vector(0,1){7}}
  \put(4.3,7){\vector(0,-1){7}}
  \put(4.8,4){\makebox(0,3)[l]{allocated block}}
    
  % context pointer
  \put(-1,4){\vector(1,0){1}}
  \put(-2,4.1){context}
    
  % talloc chunk box
  \linethickness{0.5mm}
  \put(0,7){\line(1,0){4}}
  \put(0,4){\line(1,0){4}}
  \put(0,4){\line(0,1){3}}
  \put(4,4){\line(0,1){3}}
  \put(0,4){\makebox(4,3){talloc chunk}}
 \end{picture}
  \caption{Talloc context}
  \label{fig:talloc-context}
\end{figure}

Talloc context is of type |TALLOC_CTX*|. However, |TALLOC_CTX| is only an alias
for |void| and exists only for a semantical reason -- so we can differentiate
between |void*| (arbitrary data) and |TALLOC_CTX*| (talloc context).

\subsubsection{Context meta data}
Every talloc context carries along the code several information:

\begin{itemize}
  \item name - which is used in reports of context hierarchy (section
  \ref{talloc:sec:debugging}) and to simulate dynamic type system (section
  \ref{talloc:dyn-ts})
  \item size of the requested memory in bytes
  \item attached destructor - which is executed just before the memory block is
  about to be freed (section \ref{talloc:sec:destructors})
  \item references to the context
  \item parent context
  \item children contexts
\end{itemize}

\subsection{Creating a new context}
\label{talloc:subsec:new-context}

Creating a new talloc context means that we want to create a new talloc chunk
(which stores the information about this context -- especially its parent),
allocate the desirable amount of system memory and retrieve a pointer to this
memory.

Many functions exist that have the ability to create a new talloc context. This
section describes only the most fundamental ones that deal with primitive data
types and structures. Functions that are specialized in strings and arrays are
described later in this text (sections \ref{talloc:sec:strings} and
\ref{talloc:sec:arrays}).

We can separate these functions to four categories: those that allows us to
set a name of the context, functions that creates a zero-length context and type
safe and type unsafe functions.

All of these functions share the following properties:
\begin{itemize}
  \item it returns new talloc context or |NULL| if the system is out of memory,
  \item the first parameter is a talloc context which serves as a parent of
  the new context,
  \item the parent context can be either an existing context or |NULL| which
  will create a new top level context.
\end{itemize}

\subsubsection{Contexts with custom name}

Every talloc context has a name. This name can be used for two purposes: to
differentiate from other contexts during debugging and for dynamic type
checking.

We can create a context with a custom name using the following functions: 

\begin{funcproto}
void* talloc_named(TALLOC_CTX *ctx, size_t size,
                   const char *fmt, ...)
void* talloc_named_const(TALLOC_CTX *ctx, size_t size,
                         const char *name)
\end{funcproto}
\funclistend
Hovewer, there is not much practical use for setting a custom name. These
funtions are internally used by all of the functions below which set the name
either at the current location in the source file or at the name of the data
type.

\subsubsection{Type safe functions that creates a new context}

Type safe functions take as one of their parameters a data type we want to
create. It allocates the size that is necessary for this type and returns a new
properly casted pointer. This is useful if we want to rely on the compiler to
detect type mismatches.

Another feature of these functions is that they automatically set the name of
the context to the name of the data type. This can be used for a dynamic type
checking which is described in section \ref{talloc:subsec:type-checking}.

The appropriate functions with this behaviour are:

\begin{funcproto}
(#type)* talloc(TALLOC_CTX *ctx, #type)
(#type)* talloc_zero(TALLOC_CTX *ctx, #type)
\end{funcproto}
\funclistend
The difference between |talloc()| and |talloc_zero()| is that the later ensures
the whole new memory space to be initialized with zeros.

\begin{lstlisting}[caption={talloc() and talloc_zero()},label=lst:talloc_zero]
struct user *user = talloc(ctx, struct user);
if (user == NULL) {
  return ENOMEM;
}

/* initialize to default values */
user->uid = 0;
user->name = NULL;
user->num_groups = 0;
user->groups = NULL;

/* or we can achieve the same result with */
struct user *user_zero = talloc_zero(ctx, struct user);
if (user_zero == NULL) {
  return ENOMEM;
}
\end{lstlisting}

\subsubsection{Type unsafe functions that creates a new context}

Type unsafe functions take as a parameter directly the size we want to
allocate instead of the type and return a pointer to |void|. However this way
we loose the compiler ability to detect type mismatches. Therefore we should
avoid using these functions unless we really want to retrieve a |void*| (e.g.
reading binary data from a file).

This type of functions also sets the name of the context to a current location
in the source file where the function is called.

The following functions are type unsafe variants of |talloc()| and
|talloc_zero()|:

\begin{funcproto}
void* talloc_size(TALLOC_CTX *ctx, size_t size)
void* talloc_zero_size(TALLOC_CTX *ctx, size_t size)
\end{funcproto}
\funclistend
There is also one very interesting macro |talloc_ptrtype(ctx, ptr)| that may or
may not be type unsafe depending on the compiler. If the compiler is GCC of
version greater or equal to 3, it is type safe (it uses the |__typeof__|
feature of this compiler). It is type unsafe otherwise.

It is a wrapper around |talloc_size()|, therefore the name of the context will
be the location in the source file where the macro is used. This does not depend
on whether it will be type safe or type unsafe during the compilation.

The usage can be as follows:

\begin{lstlisting}[caption={talloc_ptrtype(ctx, ptr)},label=lst:talloc_ptrtype]
struct foo *foo = talloc_ptrtype(ctx, foo);

if (foo == NULL) {
  return ENOMEM;
}

\end{lstlisting}

\subsubsection{Zero-length contexts}

Zero-length context is basically a context without any special semantical
meaning. We can use it the same way as any other context, the only difference is
that it does not contain any data and therefore it is strictly of type
|TALLOC_CTX*|. It is often used in case we want to aggregate several data
structures under one parent (zero-length) context.

\begin{funcproto}
TALLOC_CTX* talloc_init(const char *fmt, ...)
\end{funcproto}
\begin{funcdesc}
Creates a new top level zero-length context with a custom name.
\end{funcdesc}
\begin{funcproto}
TALLOC_CTX* talloc_new(TALLOC_CTX *ctx)
\end{funcproto}
\begin{funcdesc}
Creates a new zero-length context as a child of |ctx|. The name of the context
will be the current location in the source file prefixed with
\lstinline[showspaces=true]{"talloc_new: "}.
\end{funcdesc}

\begin{lstlisting}[caption={talloc_new()},label=lst:talloc_new]
TALLOC_CTX *ctx = NULL;
struct foo *foo = NULL;
struct bar *bar = NULL;

/* new zero-length top level context */
ctx = talloc_new(NULL);
if (ctx == NULL) {
  return ENOMEM;
}

foo = talloc(ctx, struct foo);
bar = talloc(ctx, struct bar);
\end{lstlisting}

\subsection{Freeing a context}
\label{talloc:subsec:free-context}

There are two functions defined that deal with deallocating a context. Both
take memory context as their argument and if this context is |NULL| then no
action is performed.

\begin{funcproto}
int talloc_free(TALLOC_CTX *ctx)
\end{funcproto}
\begin{funcdesc}
  Deallocates memory occupied by the context and recursively frees its 
  children as well. The returned value is |0| on success, |-1| if |ctx| is NULL
  or if the destructor\footnote{More information on destructors is in section
  \cmplref{talloc:sec:destructors}} attached to this context fails.
\end{funcdesc}
\begin{funcproto}
void talloc_free_children(TALLOC_CTX *ctx)
\end{funcproto}
\begin{funcdesc}
  Frees only the children of the context.
\end{funcdesc}
\funclistend
Besides these two functions we can find useful a macro named |TALLOC_||FREE|
which is defined as:

\begin{lstlisting}[caption={TALLOC_FREE(ctx)},label=lst:TALLOC_FREE]
#define TALLOC_FREE(ctx) do { \
  talloc_free(ctx);           \
  ctx = NULL;                 \
} while(0);
\end{lstlisting}

%- talloc_autofree_context
%- TALLOC_FREE
%- talloc_free - 0 or 1 parent
%- talloc_free_children
%- talloc_unlink - more parents

%-  talloc_set_destructors

\subsection{Changing the parent of a context}
\label{talloc:subsec:stealing}

Talloc has the ability to change the parent of a talloc pointer to another
context. This is necessary if we want the pointer to outlive the context it is
created on.

This is frequently used in functions that return data which has been created on
an inner context. The common usage of this technique is described in section
\cmplref{talloc:subsec:function-use-own-context}.

\begin{funcproto}
void *talloc_steal(TALLOC_CTX *ctx, const void *ptr)
\end{funcproto}
\begin{funcdesc}
  Changes the parent of the |ptr| to |ctx| and returns |ptr|.
\end{funcdesc}
\begin{funcproto}
void *talloc_move(TALLOC_CTX *ctx, const void **ptr)
\end{funcproto}
\begin{funcdesc}
  This function is a wrapper around |talloc_steal()|. It will assing |NULL| into
  |*ptr| after the parent of |ptr| is changed. Returns the same pointer as
  |talloc_steal()| would return.
\end{funcdesc}

%- talloc_reparent
%- talloc_steal
%- talloc_move

\subsection{Managing references}
\label{talloc:subsec:references}

\subsection{Decreasing number of malloc() calls}
\label{talloc:subsec:pool}

%Allocation of a new memory space is usually an expensive kernel space
%operation.

Allocation of a new memory is an expensive kernel space operation and large
programs can contain thousands of calls of |malloc()| for a single operation.
This can pour out into an unnecessary slowdown of the application. Number of
|malloc()| calls can be decreased by using a memory pool.

Memory pool\footnote{An example of a simple memory pool can be found in appendix
@TODO} is a preallocated memory space of a fixed size. If we need to allocate
new data we can simply take the desired amount from the pool instead of
requesting new memory from the system. Hovewer, this requires some special
behaviour of a programmer.

The talloc library contains its own implementation of a memory pool and the
great thing about this is that it is completely transparent for the programmer.
The only thing that needs to be done is an initialization of a new pool context
using |talloc_pool()|\footnote{\lstinline{TALLOC_CTX *
talloc_pool(TALLOC_CTX *ctx, size_t size)}}. This context can be used in the
same way as any other context.

Talloc pool context has the following properties:

\begin{itemize}
  \item if we are allocating data on a pool context, it takes the desirable
  amount of space from the pool,
  \item if the context is a descendant of the pool context, it takes the space
  from the pool as well,
  \item if the pool is out of memory, it creates a new, normal context,
  \item if we change the parent of a child of talloc pool to a parent that is
  outside of this pool, the whole pool memory will not be freed until the child
  is freed.
\end{itemize}

\begin{lstlisting}[caption={Talloc pool},label=lst:talloc_pool]
/* allocate 1KiB in a pool */
TALLOC_CTX *pool_ctx = talloc_pool(NULL, 1024);

/* take 512B from the pool, 512B left */
void *ptr = talloc_size(pool_ctx, 512);

/* 1024B > 512B, this will create new talloc chunk */
void *ptr2 = talloc_size(ptr, 1024);

/* the pool still contains 512 free bytes
 * this will take 200B from them */
void *ptr3 = talloc_size(ptr, 200);

/* this will destroy context 'ptr3' but the memory
 * is not freed, the available space in the pool
 * will increase to 512B */
talloc_free(ptr3);

/* this will free memory taken by 'pool_ctx'
 *  and 'ptr2' as well */
talloc_free(pool_ctx);
\end{lstlisting}

%- talloc_get_type
%- talloc_get_type_abort