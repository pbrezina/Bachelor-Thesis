\section{Dynamic type system}
\label{talloc:dyn-ts}

Generic programming in the C language is very difficult. There is no inheritance
nor templates known from object oriented languages. There is no dynamic type
system. Therefore, generic programming in this language is usually done by
type-casting a variable to |void*| and transferring it through a generic function
to a specialized callback as illustrated on the next listing.

\begin{lstlisting}[caption={Generic programming pattern},label=lst-generic-prg]
void generic_function(callback_fn cb, void *pvt)
{
  /* do some stuff and call the callback */
  cb(pvt);
}

void specific_callback(void *pvt)
{
  struct specific_struct *data;
  data = (struct specific_struct*)pvt;
  /* ... */
}

void specific_function()
{
  struct specific_struct data;
  generic_function(callback, &data);
}
\end{lstlisting}

\noindent
Unfortunately, the type information is lost as a result of this type cast. The
compiler cannot check the type during the compilation nor are we able to do it
at runtime. Providing an invalid data type to the callback will result
in unexpected behaviour (not necessarily a crash) of the application. This
mistake is usually hard to detect because it is not the first thing on the mind
of the developer.

As we already know, every talloc context contains a name. This name is available
at any time and it can be used to determine the type of a context even if we
lose the type of a variable.

\subsection{Typed context}

Although the name of the context can be set to any arbitrary string, the best
way of using it to simulate the dynamic type system is to set it directly to the
type of the variable.

It is recommended to use one of |talloc()| and |talloc_array()| (or its
variants) to create the context as they set its name to the name of the type
automatically.

If we have a context with such a name, we can use two similar functions that do
both the type check and the type cast for us:

\begin{funcproto}
(#type)* talloc_get_type(TALLOC_CTX *ctx, #type)
(#type)* talloc_get_type_abort(TALLOC_CTX *ctx, #type)
\end{funcproto}
\funclistend
Both functions take as the first parameter a valid talloc context and as the
second parameter the type which the context is supposed to be. If the provided
type equals to the context name, they return a properly-cast pointer to
this type.

They differ in the other scenario -- when the context name is different. The
first one simply returns |NULL|. The other one logs this violation into a talloc
log (see section \ref{talloc:sec:debugging}) and immediately calls an abort
function. By default it kills the application with |abort()| but this behaviour
can be modified by setting our own abort function with:

\begin{funcproto}
void talloc_set_abort_fn(
  void (*abort_fn)(const char *reason)
)
\end{funcproto}

\subsection{Arbitrary name check}

If we want to check the context against some custom name, we can do this using:

\begin{funcproto}
void* talloc_check_name(TALLOC_CTX *ctx, const char *name)
\end{funcproto}
\funclistend
The behaviour is the same as of |talloc_get_type()| but it does not perform the
type cast of the returned pointer.

\subsection{Examples}

Below given is an example of generic programming with talloc -- if we provide
invalid data to the callback, the program will be aborted. This is a sufficient
reaction for such an error in most applications.

\begin{lstlisting}[caption={Dynamic type system \#1}]
void foo_callback(void *pvt)
{
  struct foo *data = talloc_get_type_abort(pvt, struct foo);
  /* ... */
}

int do_foo()
{
  struct foo *data = talloc_zero(NULL, struct foo);
  /* ... */
  return generic_function(foo_callback, data);
}
\end{lstlisting}

\noindent
But what if we are creating a service application that should be running for the
uptime of a server? We may want to abort the application during the development
process (to make sure the error is not overlooked) but try to recover from the
error in the customer release. This can be achieved by creating a custom abort
function with a conditional build.

\begin{lstlisting}[caption={Custom abort function}]
void my_abort(const char *reason)
{
  fprintf(stderr, "talloc abort: %s\n", reason);
#ifdef ABORT_ON_TYPE_MISMATCH
  abort();
#endif
}

void init()
{
  talloc_set_abort_fn(my_abort);
}
\end{lstlisting}

\noindent
The usage of |talloc_get_type_abort()| would be then:

\begin{lstlisting}[caption={Sample output}]
init();

TALLOC_CTX *ctx = talloc_new(NULL);
char *str = talloc_get_type_abort(ctx, char);
if (str == NULL) {
  /* recovery code */
}
/* talloc abort: ../src/main.c:25: Type mismatch: 
   name[talloc_new: ../src/main.c:24] expected[char] */
\end{lstlisting}

% \subsection{Manipulating with context name}
%
% \begin{funcproto}
% const char* talloc_set_name(TALLOC_CXT *ctx,
%                             const char *fmt, ...)
% void talloc_set_name_const(TALLOC_CTX,
%                            const char *name)
% void talloc_set_type(TALLOC_CTX *ctx, #type)
% \end{funcproto}
% \funclistend
