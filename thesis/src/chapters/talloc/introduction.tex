Talloc is a hierarchical, reference counted memory pool system with destructors.
It is built atop the standard C library and it defines a set of utility
functions that altogether simplifies allocation and deallocation of data,
especially large structures that contains many dynamically allocated elements
such as strings and arrays.

The main goals of this library are: removing the needs for creating a cleanup
function for every complex structure, providing logical organization of
the allocated memory blocks and lowering the possibility of creating memory
leaks in the long running applications. All of this is achieved by a
hierarchical structure of talloc contexts where deallocating one context will
end up freeing all its descendants as well.

\subsubsection{Main features}
\begin{itemize}
  \item Open source project
  \item Hierarchical memory model
  \item Natural projection of data structures into the memory
  \item Simplifies memory management of large data structures
  \item Automatic execution of a destructor before the memory is freed
  \item Simulates dynamic type system
  \item Transparent memory pool
\end{itemize}