Talloc is a hierarchical, reference counted memory pool system with destructors.
It is built atop the standard C library and it defines a set of utility
functions that altogether simplifies allocation and deallocation of data,
especially for complex structures that contain many dynamically allocated
elements such as strings and arrays.

The main goals of this library are: removing the needs for creating a cleanup
function for every complex structure, providing a logical organization of
allocated memory blocks and reducing the likelihood of creating memory leaks in
long-running applications. All of this is achieved by allocating memory in a
hierarchical structure of talloc contexts such that deallocating one context
will recursively free all of its descendants as well.

\subsubsection{Main features}
\begin{itemize}
  \item Open source project
  \item Hierarchical memory model
  \item Natural projection of data structures into memory space
  \item Simplifies memory management of large data structures
  \item Automatic execution of a destructor before the memory is freed
  \item Simulates dynamic type system
  \item Transparent memory pool
\end{itemize}
