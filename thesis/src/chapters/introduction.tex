\chapter{Introduction}

Although the usual amount of the main memory has grown dramatically over the
decades, it is still one of the most critical system resources. Therefore a good
memory management is very important, especially in programs that are expected to
be running permanently. This no longer applies only to drivers and server
applications but also more and more desktop programs are used to be running
for the computer uptime (web browsers, instant messaging clients, gaming
platforms, etc.). In such programs a proper memory management is very important
because even a small memory leak that occurs periodically may end up in a much
bigger memory consumption after several hours of the application runtime. Even
though many popular programming languages contain garbage collector that takes
the memory management off the developer's shoulders, many programs are still
being written using a language such as C where the heap memory is completely
in the hands of a programmer.

Managing memory in the C language is a very difficult task. It is mainly due to
the fact that the standard library does not provide any tools for simple
deallocation of complex structures. The standard approach is to write a
specific free function for every custom data type that goes down through the
elements tree and frees them one by one from the bottom to top. The problem is
that it takes many lines of code and it is very likely that we will forget to
free something. Over the years several libraries has been developed that aims
to make the memory management in C easier and more error proof. [other libs]

Talloc is a very mature and easy to use memory allocator written in C, developed
and maintained by the Samba\footnotemark[1] team. Since it was released in
[year] it has become the main memory allocator in a few projects, including Samba,
SSSD\footnotemark[2] and nfsim\footnotemark[3]. Unfortunately the wider
expansion of this library is being blocked by the lack of tutorial and a
comprehensive description of its properties and behaviour.

\footnotetext[1]{Windows interoperability suite: 
\url{http://www.samba.org}}
\footnotetext[2]{System Security Services Daemon:
\url{https://fedorahosted.org/sssd}}
\footnotetext[3]{Netfilter simulation environment:
\url{http://ozlabs.org/~jk/projects/nfsim}}

The goal of this thesis is to create a comprehensive description of the talloc
features with code samples and a summary of the best practices of its usage
which will speed up the initial training of a new developer of the SSSD. The
thesis also aims to create a tutorial that will be published on the library
homepage and that will hopefully help the talloc to spread to many more projects.

My primary sources of knowledge are the talloc source codes, my and my
colleagues experience from the development of the SSSD and discussions with the
authors of this library.

% - why we have to care about memory consumption
% - something about memory leaks
% - difficulties of the C language
% - libraries: 
%  - libapr memory pools (apache runtime library)
% 
% - it is developed and maintained by Samba team
% - created for Samba, version ?, since ?
% - widely used in Samba, SSSD and ?
% - documentation is very short and incomplete 
% - it is not very spread due to lack of a documentation and tutorials
% - goals:
%  - correct documentation
%  - create tutorial
%  - describe how it works
%  - summarize best practices
% - sources of knowledge:
%  - existing documentation
%  - talks with authors
%  - source codes
%  - experience from SSSD development
% 
% Tevent is a very mature and easy to use event library written in C, developed
% and maintained by the Samba team.
% Unfortunately there is very little documentation available, which makes tevent
% difficult to learn. There are no tutorials, nor examples; there is only a small
% API reference on the Samba web page so one must learn the usage from very
% complex programs like Samba or SSSD. Therefore, the purpose of this thesis is
% to provide a complete set of code samples and tutorials – from simple library
% initialization to complex nested asynchronous events.
% 
% This thesis consists of three parts. The first part is an overview of event
% libraries concepts and describes how they work under the hood. The second part
% contains a description of, and tutorials for, the talloc library – the
% underlying memory allocator that is widely used in tevent. And finally the part
% three focuses on the tevent library itself.