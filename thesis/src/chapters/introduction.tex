\chapter{Introduction}

Although the usual amount of the main memory has grown dramatically over the
decades, it is still one of the most critical system resources. Therefore, a
good memory management is very important, especially in programs that are
expected to be running permanently. This no longer applies only to drivers and
server applications but also more and more desktop programs are used to be
running for the computer uptime (web browsers, instant messaging clients,
gaming platforms, etc.). In such programs a proper memory management is very
important because even a small memory leak that occurs periodically may result
in a much bigger memory consumption after several hours of the application
runtime. Even though many popular programming languages contain a garbage
collector that takes the memory management off the developer's shoulders, many
programs are still written using a language such as C\cite{CReference} where the
heap memory is completely in the hands of the programmer.

\section{Managing memory in C}

Managing memory in the C language is a very difficult task. It is mainly due to
the fact that the standard library does not provide any tools for simple
deallocation of complex structures. The standard approach is to write a
specific free function for every custom data type that goes down through the
elements tree and frees them one by one from the bottom to the top. The problem
is that it takes many lines of code and it is very likely that something is
forgotten to be freed. Over the years, few libraries has been developed that
aim to make the memory management in C easier and more error proof. Examples of
this kind of libraries are memory pools from the APR\footnote{Apache Portable
Runtime: \url{http://apr.apache.org/}} and halloc\footnote{Hierarchical memory
allocator: \url{http://swapped.cc/halloc}}. Another library, which is
originally based upon halloc\cite{MailListOrigin}, is talloc.

Talloc\footnote{Talloc memory allocator: \url{http://talloc.samba.org}} is a
very mature and easy to use memory allocator written in C, developed and
maintained by the Samba\footnote{Windows interoperability suite:
\url{http://www.samba.org}} team. Since its first public release in
2009\cite{MailListOrigin} it has become the main memory allocator in a few
projects, including Samba, SSSD\footnote{System Security Services Daemon:
\url{https://fedorahosted.org/sssd}} and nfsim\footnote{Netfilter simulation
environment: \url{http://ozlabs.org/~jk/projects/nfsim}}. Unfortunately, the
wider expansion of this library is being blocked by the lack of tutorials and a
comprehensive description of its properties and characteristics.

\section{Goals of this thesis}

The goal of this thesis is to create a comprehensive description of the talloc
features with code samples and a summary of the best practices of its usage,
which can speed up the initial training of new developers of the SSSD. The
thesis also aims to create a tutorial that will be published on the library
homepage and that will hopefully help the talloc to spread to many more projects.
