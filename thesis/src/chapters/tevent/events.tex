\section{Registering events}

Tevent supports several event types: timers,  signals, file descriptor events
(readiness for reading or writing) and immediate events. Each of these
are supported by all built-in event backends.

Although we have to use different functions to set up a new event handler for
each kind of event, their usage and prototypes are very similar. In order to
register an event handler, we have to provide five pieces of information:

\begin{itemize}
  \item the event context where the event handler will be registered,
  \item a talloc memory context which will be used as a parent for this handler
  meta data,
  \item private data that will be passed to the handler,
  \item the event handler itself,
  \item plus some event specific parameters (time, signal number, file
  descriptor, \ldots).
\end{itemize}

The functions return a pointer to an event specific structure or |NULL| on
error. This structure is a talloc context. We must not set a destructor on this
context because it already contains one -- when we call |talloc_free()| on this
pointer the event handler will be automatically unregistered from the event
context.

\subsection{Timers}

Timers are events that are triggered after a specified time span has left. The
important thing to know is that this kind of event is not triggered exactly
when the selected time is reached but as soon as possible after it is reached.
Therefore it can not be used in situations where the accuracy is critical.

To create a timed event, we will use:

\begin{funcproto}
struct tevent_timer* tevent_add_timer(
             struct tevent_context *event_ctx,
             TALLOC_CTX *mem_ctx,
             struct timeval next_event,
             tevent_timer_handler_t handler,
             void *private_data)
\end{funcproto}
\funclistend
The |next_event| parameter contains the absolute time when this event should be
fired. Tevent contains several utility functions\cite{TeventDocTimeval}
that helps us with the creation of |timeval| structure. In most cases we will
want to use an offset that is relative to the current time. This can be achieved
with:

\begin{funcproto}
struct timeval tevent_timeval_current_ofs(uint32_t secs,
                                          uint32_t usecs)
\end{funcproto}
\funclistend
The event handler for timers has the following prototype:

\begin{funcproto}
void handler(struct tevent_context *event_ctx,
             struct tevent_timer *timer,
             struct timeval current_time,
             void *private_data)
\end{funcproto}
\funclistend
The |current_time| parameter is a little tricky. As its name suggests it
contains the absolute time when the handler has been called. But only if we have
provided a non-zero time in |tevent_add_timer()|. It will contain zero
otherwise. However we should use an immediate event for such cases.

Every timer in tevent is only a one time event (the |tevent_timer| structure is
freed once the handler returns). If we want it to be executed periodically we
have to call |tevent_add_timer()| in the handler again.

In our example we will set a periodical timer that will be fired every sixty
seconds.

\begin{lstlisting}[morekeywords={tevent_add_timer},name=PeriodicalUpdateDB]
struct tevent_timer* set_periodical_update_db(
                  struct tevent_context *ev,
                  TALLOC_CTX *mem_ctx,
                  void *pvt)
{
  struct timeval tv;
  tv = tevent_timeval_current_ofs(60, 0);
  return tevent_add_timer(ev, mem_ctx, tv,
                          periodical_update_db, pvt);
}
\end{lstlisting}
\begin{lstlisting}[caption={Periodical update of a database},firstnumber=auto,
morekeywords={set_periodical_update_db},name=PeriodicalUpdateDB]
void periodical_update_db(struct tevent_context *event_ctx,
                          struct tevent_timer *timer,
                          struct timeval current_time,
                          void *pvt)
{
  struct tevent_timer *timer = NULL;
  struct update_ctx *ctx = NULL;
  ctx = talloc_get_type_abort(pvt, struct update_ctx);

  update_db(ctx);
  
  timer = set_periodical_update_db(ev, ctx->mem_ctx, ctx);
  if (timer == NULL) {
    /* error */
  }
}
\end{lstlisting}

\subsection{Signals}

\subsection{File descriptor events}

\subsection{Immediate events}