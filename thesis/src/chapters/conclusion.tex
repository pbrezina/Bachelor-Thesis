\chapter{Conclusion}

Although I have used the available documentation as a starting point, my primary
source of knowledge was the talloc source code. This gave me a very deep
understanding of the library and it also taught me some nice C and GCC tricks.

As a result of my work I have managed to send several patches to the
samba-technical mailing list, which improve the talloc doxygen documentation.
Those patches have already become a part of the Samba upstream and the new
documentation is available online on the project homepage since 18th April 2012.

I have also created the tutorial for the talloc library. This tutorial is
written in the doxygen\footnote{www.doxygen.org} format, thus it can nicely
fit into the existing online documentation. It has been published on the project
homepage.

As one of the developers of the SSSD, I have introduced to the team a few
interesting features that can give the product a better performance (talloc
pools, |buffer| version of string concatenation functions) or that could help
us to avoid type mistakes (custom abort function, |talloc_get_type_abort()|).

I have compiled the best practices from my one-year experience with the
development of the SSSD and from many discussions with experienced Red Hat
developers and Samba team members.

As an open source enthusiast, I am glad that I have been able to contribute to
this great open source project and help it to become more accessible.
