\chapter{Conclusion}

Although I have used the available documentation as a starting point, my primary
sources of knowledge were the talloc source codes. This gave me a very deep
understanding of the talloc library and it also taught me some nice C and GCC
tricks.

As a result of my work I managed to send several git-formatted patches to the
samba-technical mailing list that improves the talloc doxygen documentation.
Those patches have already become a part of the Samba upstream and the new
documentation is available online on the project homepage since 18th April 2012.

[doxygen tutorial]

As one of the developers of the SSSD I have introduced to the team few
interesting features that can give the product a better performance (talloc
pools, |buffer| version of string concatenation functions) or that should help
us avoiding type mistakes (custom abort function, |talloc_get_type_abort()|).

The best practises were forged from my one-year experience from the development
of the SSSD and from many discussions with experienced Red Hat developers and
Samba team members.

As an open source enthusiast I am glad that I was able to contribute to this
great open source project and help it to become more accessible.


% - sources of knowledge
% - talloc documentation patches
% - sssd suggestions
% - doxygen tutorial
% - talloc will be more accessible
% 
% <jhrozek> kdyby ses chtel vic pochvalit tak muzes rict ze ti na best practices pomahali "experienced Red Hat developers" a ze jsi s nima musel komunikovat i pres casovy posun
% <pbrezina> jo to by šlo
% <pbrezina> a co dát do závěru?
% <jhrozek> hm, tohle by se mozna spis hodilo do zaveru
% <jhrozek> no obecne bych tam rekl ze diky tve praci se stane talloc "more accessible", ze ma poradnou dokumentaci a ze jsi ho cely prolezl
% <jhrozek> a ze bude mit dokumentaci i upstream
% <jhrozek> jo a to ze ti to cte Andreas bych klidne podal tak ze ti "Samba team member" udelal recenzi
% <jhrozek> protoze Andreas prece jen ma e-mail @samba.org
% <jhrozek> jo a ze jsi navrhnul nekolik zlepseni SSSD projektu
% <jhrozek> neva ze se Simovi nelibily